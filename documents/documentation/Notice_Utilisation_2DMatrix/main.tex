\documentclass[oneside,12pt]{scrartcl}
\usepackage[utf8]{inputenc}
\usepackage{graphicx}
\usepackage{eso-pic}
\usepackage{caption}
\usepackage{subcaption}
\usepackage{minted}
\usepackage{parskip}
\usepackage{xcolor}
\usepackage{silence}
\usepackage[hidelinks]{hyperref}
\usepackage{arydshln}
\usepackage{mathtools}
\usepackage{comment}





\newcommand*{\escape}[1]{\texttt{\textbackslash#1}}
\newcommand{\insererText}{\textcolor{red}{[INSERER TEXTE ICI]}}
\newcommand{\insererTextT}[1]{\insererText\ \textcolor{red}{#1}}
\newcommand{\insererImage}{\textcolor{red}{[INSERER IMAGE ICI]}}
\newcommand{\HRule}{\rule{\linewidth}{0.5mm}} % Defines a new command for the horizontal lines, change thickness here


\usepackage{xpatch,letltxmacro}
\LetLtxMacro{\cminted}{\minted}
\let\endcminted\endminted
\xpretocmd{\cminted}{\RecustomVerbatimEnvironment{Verbatim}{BVerbatim}{}}{}{}

\renewcommand{\thesection}{\Roman{section}} %redefini apparence nom sections (affichage chiffres romains)

\AddToShipoutPicture*
{
\put(235,100) %(x=100, y=100)
 {\includegraphics[scale=0.3]{imgs/logo-univ-rouen.png}}
}
\begin{document}

\begin{titlepage}

\center % Center everything on the page

%----------------------------------------------------------------------------------------
%	TITLE SECTION
%----------------------------------------------------------------------------------------

%\HRule \\[0.4cm]
%{ \huge \bfseries Interaction between Artificial Intelligence and Computer Architecture}\\[0.4cm] % Title of your document
%\HRule \\[1.5cm]

\HRule \\[0.4cm]
{ \huge \bfseries Protocole 2D-Doc et Algorithmes}\\[0.4cm] % Title of your document
%{ \huge \bfseries Leveraging Artificial Intelligence Algorithms \\ to Design Hardware Prediction Mechanisms}\\[0.4cm] % Title of your document
\HRule \\[1.5cm]

\textsc{\huge \textbf{Documentation Technique}}\\[1.5cm]

\textsc{\Large \textbf{Université de Rouen Normandie \\[0.07cm] UFR des Sciences et Technique \\[1,5cm] Master 1 Informatique\\[0.5cm] Projet Datamatrix M1 SSI 2021-2022}}


%----------------------------------------------------------------------------------------
%	AUTHOR & ADVISOR(S) SECTION
%----------------------------------------------------------------------------------------

\vspace{3cm}

\begin{center} \large
\Large \emph{Auteur:}\\[0.1cm]
\large BELABDOUN Billal\\
\end{center}
\vspace{1cm}

%----------------------------------------------------------------------------------------
%	DATE SECTION
%----------------------------------------------------------------------------------------

{\large Mai 2022}\\[1cm]
%----------------------------------------------------------------------------------------

\vfill % Fill the rest of the page with whitespace

\end{titlepage}
\thispagestyle{empty} %pour retirer la 1ere page de la pagination

\newpage

\begin{center}
\textsc{\Large \textbf{Révision du document}}
\end{center}
\vspace{2cm}
\begin{center}
\begin{tabular}{|c|c|c|}
    \hline
  N° de version & Date de modification & Modifications apportées  \\
    \hline
    1 & 19/04/2022 & Création du document\\
    \hline
\end{tabular}
\end{center}


\newpage
\pagenumbering{arabic} %remet la pagination pour le sommaire
{
\renewcommand*\contentsname{Sommaire}
\tableofcontents
}

\clearpage

\section{Préliminaires}
Pour rappel, l'ensemble des termes utilisés dans cette notice est défini dans le document de \texttt{Spécification Technique des Besoins} (STB)

Dans le cadre de notre projet, nous avons dû mettre en place deux serveurs pour héberger nos différents outils. Le premier serveur contient la partie \texttt{Gestion} avec l'ensemble des outils administratifs et le second, la partie \texttt{Scan}.
Les deux serveurs sont accessibles en \texttt{HTTPS} avec les adresses :
\begin{minted}{bash}
# Serveur de Gestion
https://srv-dpi-proj-datamatrix-srv1.univ-rouen.fr/

# Serveur de Scan
https://srv-dpi-proj-datamatrix-srv2.univ-rouen.fr/
\end{minted} 
\vspace{1cm}

En ce qui concerne le serveur de Gestion, il faut installer un certificat délivré et signé par l'autorité \texttt{2DMP} (2DMatrix Project) afin d'accéder à l'espace de Super Administrateur. En effet, le serveur utilise un système d'authentification forte \texttt{SSL/TLS} avec vérification de certificat client.

\small\textit{\textbf{Remarque} : Pour une utilisation depuis l'extérieur, il faut avoir ajouté le \texttt{VPN} pour être connecté sur le réseau du Département Informatiques de l'UFR Sciences et Techniques du Madrillet afin de pouvoir accéder à ces serveurs. De plus, il faut ajouter le certificat du département pour que la connexion soit considéré comme sécurisée.}
\section{Authentifications}
Nous allons commencer par le serveur de \texttt{Gestion}. En effet, nous allons générer des 2D-Doc dont nous nous servirons avec l'outil \texttt{2DMatrix-Scan}.

En se connectant sur le serveur de \texttt{Gestion}, nous allons apparaître sur une page de connexion pour l'ensemble des outils de ce serveur.

\subsection{Super Administrateur}

La première étape consiste à se connecter en tant que \texttt{Super Administrateur} en cliquant sur l'image de certificat. On vous demande ensuite de valider l'authentification avec un certificat client (nommé \texttt{SUPADM}) délivré par notre autorité \texttt{2DMP}. Une fois validez, vous serez redirigé vers l'outil \texttt{2DMatrix-Manage} dans la partie réservé au Super Administrateur. Veuillez voir la sous-section \hyperref[managesupadm]{\texttt{Super Administrateur}} de la section \texttt{2DMatrix-Manage} pour comprendre le fonctionnement de cette page. 

\subsection{Administrateur Organisme}

Après avoir créé un (ou plusieurs) organisme, un compte Administrateur Organisme aura été généré au moment de la création. Vous pouvez donc vous connecter avec les identifiants associés à ce compte. Vous serez automatiquement redirigé vers l'outil \texttt{2DMatrix-Manage} dans la partie des \texttt{Administrateurs Organisme}. Veuillez voir la sous-section \hyperref[manageadmorga]{\texttt{Administrateur Organisme}} de la section \texttt{2DMatrix-Manage} pour comprendre le fonctionnement de cette page. 

\subsection{Utilisateur Organisme}

Après avoir créé un ou plusieurs compte utilisateur organisme, vous pouvez, comme pour les administrateurs organismes, simplement entrer vos identifiants dans le formulaire de connexion. Vous serez automatiquement redirigé vers l'outil \texttt{2DMatrix-Create}. Veuillez voir la section \hyperref[create]{\texttt{2DMatrix-Create}} pour comprendre le fonctionnement de cette page. 
\section{2DMatrix-Manage}
\subsection{Partie super administrateur}\label{managesupadm}
\subsubsection{Création d'un Organisme}
L'une des premières étapes est de créer un Organisme, qui servira à générer les 2D-Doc authentifiant des documents officiels comme les diplômes par exemple.

Avec ce formulaire, on crée un organisme mais également un compte administrateur associé à ce dernier :
\begin{center}
    Organisme
\end{center}
\begin{itemize}
    \item \texttt{Identifiant organisme} : Un identifiant unique sur 4 caractères stocké en majuscule.
    \item \texttt{Nom de l'organisme} : Le nom complet de l'organisme.
    \item \texttt{Pays de l'organisme}
    \item \texttt{Ville de l'organisme}
\end{itemize}
\begin{center}
    Compte administrateur organisme
\end{center}
\begin{itemize}
    \item \texttt{Nom d'utilisateur}: Le nom d'utilisateur pour l'administrateur de l'organisme créé
    \item \texttt{Mot de passe} : Le mot de passe de ce compte administrateur organisme
    \item \texttt{Prénom} : Le prénom de l'administrateur organisme. 
    \item \texttt{Nom} : Le nom de l'administrateur organisme.
\end{itemize}

\subsubsection{Affichage des Organismes}
Les organismes existant sont affichés sous forme de vignettes affichant les informations principales de ces derniers :

\begin{itemize}
    \item \texttt{Identifiant organisme}
    \item \texttt{Nom de l'organisme}
    \item \texttt{Pays de l'organisme}
    \item \texttt{Ville de l'organisme}
    \item \texttt{Date de création }
    \item \texttt{Nom de compte de l'administrateur organisme associé}
\end{itemize}

Une image sous forme de flèche dans chaque vignette permet \texttt{d'exporter} les informations de l'organisme en téléchargeant au format \texttt{JSON} les informations de ce dernier. Pour finir, un bouton \texttt{Exporter tous} permet de télécharger un fichier \texttt{JSON} contenant l'ensemble des organismes existants.

Vous trouverez dans la sous-section \hyperref[import]{\texttt{Importation d'un Organisme}}, la syntaxe du fichier \\d'import/export.

Vous trouverez en bas de page, deux formulaires pour la révocation et suppression d'organisme.
\subsubsection{Suppression d'un Organisme}
Le premier formulaire, permet de supprimer un organisme, c'est à dire de le retirer de la base de données mais de la laisser dans notre \texttt{TSL} afin que l'ensemble des \texttt{2D-Doc} préalablement générer par cette organisme reste valide et vérifiable par notre outil \texttt{2DMatrix-Scan}. 

Le formulaire permet de choisir dans une liste déroulante un identifiant d'organisme existant.

\subsubsection{Révocation d'un Organisme}

Ce deuxième formulaire, permet de révoquer un organisme, c'est à dire de le supprimer de la base de donnée\footnote{Reprend le comportement du formulaire précédent.} et également de notre \texttt{TSL}. Cela implique que les \texttt{2D-DOC} générés par cet organisme sont invalidé pour notre outil \texttt{2DMatrix-Scan}.

Le formulaire permet de choisir dans une liste déroulante un identifiant d'organisme existant ou un organisme déjà supprimer mais pas encore révoquer.

\subsubsection{Importation d'un Organisme}\label{import}

Le formulaire d'importation d'organisme permet d'ajouter des organisme grâce à un fichier \texttt{JSON} avec la même syntaxe que les organismes exportés. 

\begin{center}
    Syntaxe du fichier JSON d'import/export
\end{center}

\begin{minted}{JSON}
{
    "Authorities": {
        "ID ORGANISME": {
            "ID": "ID ORGANISME",
            "ORGANIZATION_NAME": "NOM ORGANISME",
            "ORGANIZATION_COUNTRY": "FR",
            "ORGANIZATION_CITY": "VILLE ORGANISME",
            "CREATION_DATE": "YYYY-MM-DD",
            "ORGANIZATION_PKEY": "CLE PRIVEE ECDSA",
            "CERTIFICATS": "CERTIFICAT ECDSA",
            "ADMIN_ACCOUNT": {
                "ID": "IDENTIFIANT COMPTE ADMINISTRATEUR",
                "FIRST_NAME": "PRÉNOM ADMINISTRATEUR",
                "LAST_NAME": "NOM ADMINISTRATEUR",
                "USERNAME_ACCOUNT": "NOM DE COMPTE ADMINISTRATEUR",
                "PASSWORD_ACCOUNT": "HACHE DU MOT DE PASSE"
            }
        }
        "ID ORGANISME 2": {
            "..."
        }
    }
}
\end{minted}

\subsection{Partie administrateur organisme}\label{manageadmorga}

\subsubsection{Création d'un compte utilisateur organisme}
On souhaite à présent, créer un compte utilisateur organisme qui sera le compte utilisant l'outil de génération de \texttt{2D-Doc} (\texttt{2DMatrix-Create}).

Voici les informations à renseigner dans le formulaire :
\begin{itemize}
    \item \texttt{Nom d'utilisateur}
    \item \texttt{Mot de passe}
    \item \texttt{Prénom}
    \item \texttt{Nom}
\end{itemize}

Après avoir rempli les informations, cliquez simplement sur \texttt{Créer}.

\subsubsection{Affichage des utilisateurs de l'organisme}
Les utilisateurs de l'organisme auquel est rattaché le compte administrateur organisme sont affichés sous forme de vignettes affichant les informations principales de ces derniers :

\begin{itemize}
    \item \texttt{Identifiant unique du compte}
    \item \texttt{Nom d'utilisateur}
    \item \texttt{Prénom} 
    \item \texttt{Nom}
    \item \texttt{Date de création}
\end{itemize}

Une image de corbeille dans chaque vignette permet de \texttt{supprimer} le compte représenté par cette dernière.
\section{2DMatrix-Create}\label{create}

\subsection{Création d'un Datamatrix}
Le formulaire permet de générer un datamatrix, certaines valeurs de champs peuvent être choisi arbitrairement tel que le nom et le prénom mais d'autres champs doivent contenir une valeur parmi celles qui sont proposées. Les informations qui sont a renseignées dans le formulaire sont : 

\begin{itemize}
    \item \texttt{Nom} : Donnée composée de 0 à 38 caractères de type alphanumérique.
    \item \texttt{Prénom} : Donnée composée de 0 à 20 caractères de type alphanumérique.
    \item \texttt{Date de naissance} : Donnée indiquant la date de naissance.
    \item \texttt{Pays de naissance} : Donnée composée de 2 caractères de type alphanumérique 
    \\parmi la liste de pays proposée.
    \item \texttt{Genre} : Donnée composée de 1 caractère de type alphanumérique 
    \\parmi la liste de genre proposée.
    \item \texttt{Numéro du diplôme} : Donnée composée de 0 à 20 caractères de type alphanumérique
    \\parmi la liste de genre proposée. 
    \item \texttt{Niveau de diplôme} : Donnée composée de 1 caractère de type numérique
    \\parmi la liste des diplômes proposée.
    \item \texttt{Type de diplôme} : Donnée composée de 2 caractères de type alphanumérique
    \\parmi la liste de type de diplôme proposée. 
    \item \texttt{Domaine} : Donnée composée de 0 à 30 caractères de type alphanumérique.
    \item \texttt{Mention} : Donnée composée de 0 à 30 caractères de type alphanumérique.
    \item \texttt{Spécialité} : Donnée composée de 0 à 30 caractères de type alphanumérique.
    \item \texttt{Chiffrer données} : Permet de chiffrer les données afin d'être ininterprétable par les autres scans
    \item \texttt{Sélectionner un certificat} : Donnée composée d'un identifiant de certificat sur 4 caractères parmi une liste
    d'identifiant de certificat proposée.
\end{itemize}

Après avoir renseigné toutes ces informations, le bouton \texttt{Valider} permet de générer le datamatrix avec les informations renseignées et la signature conformément au protocole 2D-DOC. Il est possible d'exporter le datamatrix en cliquant sur le bouton \texttt{Télécharger}.

\subsection{Revoquer un Datamatrix}
Le menu de révocation d'un datamatrix est accessible en cliquant sur le lien \texttt{Gestion Blacklist} depuis le menu de création d'un datamatrix. Pour révoquer un datamatrix, il est nécessaire de le fournir à l'outil pour qu'il puisse récupérer directement la signature et l'ajouter dans les champs pour ajouter la signature dans la liste de révocation (ou la retirer). Vous pouvez donc ensuite choisir en cliquant sur les boutons \texttt{Ajouter} ou \texttt{Supprimer}.
\section{2DMatrix-Scan}
\subsection{Description de l'outil}
L'outil de \texttt{2DMatrix-Scan} est accessible sans authentification et permet de scanner des \texttt{QRCode} de passe sanitaire, ainsi que des \texttt{2D-Doc} générés depuis l'application \texttt{2DMatrix-Create}.
Une fois la page chargée, il est entièrement fonctionnel sans connexion.
\subsection{Scan d'un code}
Un code peut être scanné de 2 manières différentes :
\begin{itemize}
    \item à l'aide d'une caméra
    \item directement depuis un fichier image
\end{itemize}

Une fois le code scanné, le fonctionnement de l'outil est identique, peu importe la méthode utilisée.

\subsubsection{Scan depuis une caméra}
Pour pouvoir utiliser la caméra de la machine, il faut donner la permission au navigateur de s'en servir. Pour cela, il suffit de cliquer sur le bouton \texttt{REQUEST CAMERA PERMISSIONS}, puis d'autoriser dans la pop-up ouverte par le navigateur.

Si la permission est déjà donnée, il suffit de cliquer sur \texttt{START SCANNING} si la caméra n'est pas encore lancée.

\subsubsection{Scan depuis un fichier}
Pour scanner depuis un fichier, il suffit de cliquer sur \texttt{Scan an Image File} puis de sélectionner le fichier d'image voulu.

\subsection{Détail des données}
\subsubsection{Détails communs aux passes sanitaires et 2D-Doc}
Une fois le code scanné, le contenu brut du code, ainsi qu'une vue simplifiée comportant seulement quelques informations (Ex: date du passe, nom/prénom, validité, pour les passes sanitaires) sont affichés. Pour avoir accès aux détails ainsi qu'à l'explication du traitement permettant de récupérer les données claires, il suffit de cliquer sur le bouton \texttt{Afficher} en bas de page.

On retrouve dans un premier temps le détail des règles de validation du passe sanitaire/2D-Doc, suivi d'un tableau récapitulant les données contenu dans le passe sanitaire/2D-Doc scanné. Dans ce tableau, se trouvent donc les données brutes (lues directement lors du scan), ainsi que leur conversion en données claires (si nécessaire) accompagnées d'une explication.

\subsubsection{Particularité pour les passes sanitaires}
Dans le cas du passe sanitaire, les données sont converties en objet, compressées avec deflate, encodées en base45 et ne sont donc pas interprétables directement après le scan. Le détail des transformations opérées afin de pouvoir analyser le contenu du passe est afficher sous le tableau de données. Un lien permettant de s'y rendre est disponible au niveau des données brutes du passe (sous la section de scan).

\subsubsection{Particularité pour les 2D-Doc}
En cliquant sur les lignes du tableau des données contenues, la valeur brute sera surlignée dans le champ du code brut. Cela n'est possible que pour les 2D-Doc car les données d'un passe sanitaire n'apparaissent pas en clair directement après un scan.

\subsection{Modification des règles de validation d'un passe sanitaire}
Les règles de validations d'un passe sanitaire (hors authenticité des données) peuvent être modifiées. Pour cela, sous le détail des différentes règles, se trouvent des cases à cocher/champs permettant d'activer ou de désactiver certaines règles, ou bien de modifier les valeurs pour d'autres. Par exemple, il est possible de changer le nombre de doses nécessaire afin de valider les conditions sur les données.

Après avoir modifié les règles, il est nécessaire de cliquer sur le bouton \texttt{APPLIQUER} afin de relancer le processus de validation.

\subsection{Modification des données}
Les données contenues dans les passes sanitaire/2D-Doc peuvent être modifiées afin de montrer le lien entre données et signature. Pour se faire, il suffit de cliquer sur le bouton \texttt{MODIFIER DONNEES}. Cela permet de rendre les champs de données brutes éditables. Les données peuvent donc être modifier, et le passe sanitaire/2D-Doc sera régénéré lors de l'appui sur le bouton \texttt{VALIDER MODIFICATIONS}. Le processus de validation se relancera automatiquement.

\subsection{Export d'un code}
L'outil \texttt{2DMatrix-Scan} offre la possibilité de télécharger le code en cours d'utilisation (après des modifications par exemple). Le type de code généré correspond au type de code scanné. C'est à dire qu'un QRCode sera généré si c'est un passe sanitaire qui a été scanné, mais un Datamatrix sera généré si le scan a porté sur un 2D-Doc créé par l'outil \texttt{2DMatrix-Create}. Dans le cas d'un code lambda (ni passe sanitaire, ni 2D-Doc), le code généré sera, par défaut, un QRCode.




\end{document}