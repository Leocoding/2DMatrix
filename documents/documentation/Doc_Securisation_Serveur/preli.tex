\section{Préliminaire}
\subsection{Prérequis}
Nous procéderons ici à l'installation d'un serveur apache 2.4 sur une machine virtuelle tournant sous Ubuntu 21.04 .  Le gestionnaire de paquet ici est Apt, si vous êtes sous Fedora référé vous à dnf. Nous utiliserons également l'éditeur de texte \texttt{nano} (préinstallé sur ubuntu) mais il est possible d'utiliser n'importe quel autre.

\subsection{Connexion ssh}
La connexion ssh (pour Secure Shell) a pour but d'établir une connexion sécurisée (chiffrée) entre votre machine réelle (le client) et une machine distante (le serveur), ici une machine virtuelle. Cette connexion va vous permettre d'accéder à votre machine virtuelle, depuis votre shell sur votre machine réelle. \\
Par convention, dans notre document nous noterons les commandes à exécuter sur notre machine réelle sous la forme : 
\begin{minted}{bash}
user@client:~$ COMMANDE
\end{minted}
et les commandes à exécuter sur notre machine virtuelle sous la forme : 
\begin{minted}{bash}
admin@serveur:~$ COMMANDE
\end{minted}
Normalement openssh est installé par défaut sur votre machine réelle, dans le cas contraire la commande suivante le fera :
\begin{minted}{bash}
user@client:~$ sudo apt install openssh-client
\end{minted}

Pour établir une connexion ssh entre votre machine réelle et virtuelle, ouvrez un terminal sur votre machine réelle et tapez la commande suivante : 
\begin{minted}[breaklines]{bash}
ssh <user>@<ip> #remplacer <user> par le nom de l'utilisateur et <ip> par l'ip de votre machine.
#par exemple  :
user@client:~$ ssh administrateur@10.130.108.30
\end{minted}
Indiquez ensuite votre mot de passe et vous voila connecté !\\
Si vous avez changé le port par défaut, vous devrez préciser le port à la connexion. 
\begin{minted}{bash}
ssh -p <port> <user>@<ip>
#par exemple  :
user@client:~$ ssh -p 3456 administrateur@10.130.108.30
\end{minted}

Pour sécuriser d'avantage la connexion ssh, il est possible d'utiliser des clefs ssh.

Pour se faire nous allons générer une paire de clefs privée/publique à partir de la machine réelle (utilisée pour effectuer la connexion) à l'aide de la commande suivante :

    \begin{minted}{bash}
user@client:~$ ssh-keygen    
    \end{minted}

Le chemin par défaut est \texttt{$\sim$/.ssh/id\_rsa} et est automatiquement pris en compte par le démon ssh. Validez avec la touche \texttt{ENTRER}, puis laissez la passphrase vide afin quelle ne soit pas demandée à chaque connexion.

2 fichiers seront alors créés dans le répertoire \texttt{.ssh}:

    \begin{itemize}
         \item \texttt{$\sim$/.ssh/id\_rsa} qui correspond à la clef privée\\
         \item \texttt{$\sim$/.ssh/id\_rsa.pub} qui correspond à la clef publique\\
    \end{itemize}
Il ne reste plus qu'à ajouter la clef publique sur le serveur distant pour qu'il autorise la connexion:

\begin{minted}{bash}
user@client:~$ ssh-copy-id -i ~/.ssh/id_rsa.pub [-p <port>] admin@serveur
\end{minted}

Le paramètre \texttt{-i} permet de spécifier la clef publique à envoyer. Le paramètre du port est facultatif mais et nécessaire dans le cas où le port par défaut du service \texttt{ssh} a été modifié.

Une fois cette commande tapée, le mot de passe est demandé afin d'authentifier a connexion mais ne sera plus utile les prochaines fois.

On peut interdire les connexion par mot de passe en modifiant le fichier de configuration du service sshd : \texttt{/etc/ssh/sshd\_config} afin de changer la valeur de \texttt{PasswordAuthentication}:

\begin{minted}{text}
PasswordAuthentication no
\end{minted}

Un redémarrage du service est nécessaire afin d eprendre en compte les modifications:
\begin{minted}{bash}
sudo systemctl restart sshd
\end{minted}

Toute connexion via mot de passe sera dorénavant impossible et seul l'authentification par clef ssh sera acceptée.




%Peut être parler ici d'un snapshot ?

\subsection{Installation d'Apache2}
Nous allons ici voir comment installer un serveur apache sur notre VM. Pour ce faire il nous faut installer le service apache avec la commande suivante : 
\begin{minted}{bash}
admin@serveur:~$ sudo apt install apache2
\end{minted}

Une fois apache installé, voici quelques commandes pouvant vous être utiles :  
\begin{minted}[breaklines]{bash}
admin@serveur:~$ systemctl status apache2 #vous indique l'état du service apache
admin@serveur:~$ sudo systemctl reload apache2 #permet de recharger la configuration d'apache
admin@serveur:~$ sudo systemctl restart apache2 #permet de relancer apache
\end{minted}






















