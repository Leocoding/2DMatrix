\section{Introduction}
De nos jours la sécurisation des serveurs est indispensable, car les cybercriminels les ciblent en priorité. En effet, c’est l’endroit idéal pour y trouver des données sensibles. 

Lors de l'exploitation d'un serveur Web, il est important de mettre en place des mesures de sécurité pour protéger votre site et vos utilisateurs. Vous pouvez protéger vos sites Web et vos applications avec, par exemple, des politiques de pare-feu et restreindre l'accès à certaines zones avec une authentification par mot de passe. Ceci est un excellent point de départ pour sécuriser votre système, mais n'est pas suffisant. Il est important de s’assurer que des mesures de sécurité robustes sont intégrées à votre serveur et vos applications afin de maintenir un niveau de sécurité élevé. \\
Avec la pandémie et le télétravail, les problèmes de piratage ont nettement augmenté. Pourtant, il existe un certain nombre de mesures pouvant être prises pour garantir un niveau élevé de sécurité. Voici une liste non exhaustive de mesures pouvant être prises facilement afin d'augmenter la sécurité d'un serveur : 
\begin{itemize}
    \item Établir  \hyperref[POLITIQUE]{une politique de mots de passe}
    \item Sécuriser votre serveur internet avec \hyperref[HTTPS]{HTTPS}
    \item Maintenir à jour votre serveur
    \item Préparer le serveur contre les attaques par force brute (Fail2ban)
    \item Changer le port SSH par défaut
\end{itemize}